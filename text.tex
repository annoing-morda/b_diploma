\documentclass[pdftex,ptm,12pt,a4paper]{report}
\renewcommand{\baselinestretch}{1.5}
\setcounter{secnumdepth}{5}

% PDF search & cut'n'paste
\usepackage{cmap}

% Cyrillic support
\usepackage{mathtext}
\usepackage[T1,T2A]{fontenc}
\DeclareSymbolFont{T2Aletters}{T2A}{cmr}{m}{it}
\usepackage[utf8]{inputenc}

\usepackage[bottom=20mm,top=20mm,right=20mm,left=30mm,headsep=0cm,nofoot]{geometry}

\makeatletter
\renewcommand*{\ps@plain}{%
  \let\@mkboth\@gobbletwo
  \let\@oddhead\@empty
  \def\@oddfoot{%
    \reset@font
    \hfil
    \thepage
    % \hfil % removed for aligning to the right
  }%
  \let\@evenhead\@empty
  \let\@evenfoot\@oddfoot
}
\makeatother
\pagestyle{plain}
\setlength{\footskip}{30pt}

\usepackage[pdftex]{graphicx}
\usepackage{caption}
\usepackage{subcaption}
\usepackage[english, russian]{babel}
    \addto{\captionsenglish}{\renewcommand{\bibname}{Литература}}
    \addto\captionsenglish{\renewcommand{\figurename}{Рис.}}
    \addto\captionsenglish{\renewcommand{\contentsname}{Содержание}}
    \addto\captionsenglish{\renewcommand{\proofname}{Доказательство}}
\usepackage{hyperref}
\usepackage{url}
\usepackage{abstract}
\usepackage{float}
\usepackage{amsthm}
\usepackage{amssymb}
\usepackage{amsmath}
\renewcommand*{\proofname}{Доказательство}
\usepackage{indentfirst}
\usepackage{color}
\usepackage{natbib}
\usepackage{bbm, dsfont}
\usepackage{titlesec}
\usepackage{caption}

% Detect whether PDFLaTeX is in use
\usepackage{ifpdf}

% Fix links to floats
\usepackage[all]{hypcap}

%Formula packages
\usepackage{amsmath}

% Chapter
\makeatletter
\renewcommand{\@chapapp}{Часть}
\makeatother

\titleformat{\chapter}[display]
{\normalfont\huge\bfseries}
{\chaptertitlename\ \thechapter}{0.5ex}{}[]

\titlespacing*{\chapter}{0pt}{0pt}{40pt}

% Rule for Title Page
\newcommand{\HRule}{\rule{\linewidth}{0.5mm}}

\begin{document}

\begin{titlepage}
\newpage

\begin{center}{\small
МИНИСТЕРСТВО ОБРАЗОВАНИЯ И НАУКИ РОССИЙСКОЙ ФЕДЕРАЦИИ \\
\vspace{0.1cm}
ФЕДЕРАЛЬНОЕ ГОСУДАРСТВЕННОЕ АВТОНОМНОЕ  \\*
ОБРАЗОВАТЕЛЬНОЕ УЧРЕЖДЕНИЕ ВЫСШЕГО ОБРАЗОВАНИЯ \\*
"МОСКОВСКИЙ ФИЗИКО-ТЕХНИЧЕСКИЙ ИНСТИТУТ \\*
(ГОСУДАРСТВЕННЫЙ УНИВЕРСИТЕТ)" \\*
\vspace{0.1cm}
ФАКУЛЬТЕТ ИННОВАЦИЙ И ВЫСОКИХ ТЕХНОЛОГИЙ \\*
КАФЕДРА ФИЗИКО-ТЕХНИЧЕСКОЙ ИНФОРМАТИКИ \\*
\hrulefill
}
\end{center}

\begin{center}
\Large Выпускная квалификационная работа (бакалаврская работа)
\end{center}

\vspace{1em}

\begin{center}
\Large НА ТЕМУ:
\linebreak 
\textsc{\large{\textbf{ПРОВЕРКА НАЛИЧИЯ ПЕРЕСЕЧЕНИЯ ТЕЛ В ТРЕХМЕРНОМ ПРОСТРАНСТВЕ}}}
\end{center}

\vspace{4em}

\begin{flushleft}
Студент \hrulefill Морозов Дмитрий Антонович \\
Научный руководитель \hrulefill Гуревич Михаил Исаевич\\
Зав. кафедрой \hrulefill Бутко Андрей Борисович\\ 
\end{flushleft}

\vspace{\fill}

\begin{center}
МОСКВА, 2017
\end{center}

\end{titlepage}

\tableofcontents

\clearpage

\chapter{Введение}
В данной работе использованы следующие понятияи обозначения.
Под если задан $\vec{a}$, понимаем $a$ как его координатный столбец. 
Базис при этом может быть указан как нижний индекс или отдельно в тексте.


\chapter{Описание алгоритма}
\begin{enumerate}
  \item 
      \textbf{Вход алгоритма и требования к нему.}
      На вход алгоритма поступают цилиндр и параллелограмм в следующем формате.
      Цилиндр задан четырьмя векторами $\vec{o_c}, \vec{a_c}, \vec{b_c}, \vec{c_c}$ в трехмерном афинном пространстве.
      Вектор $\vec{o_c}$ задает центр одного из оснований.
      Вектор $\vec{a_c}$ задает одну из осей эллипса, лежащего в основании.
      Вектор $\vec{b_c}$ задает вторую ось эллипса, лежащего в основании.
      Вектор $\vec{с_c}$ задает направляющую цилиндра.
      Отсюда вытекают следующие требования.
      Во-первых, координаты векторов должны быть заданы в ортонормированном базисе. 
      Здесь имеется в виду, что программа считает исходный базис ортонормированным.
      Во-вторых, вектора $\vec{a_c}$ и $\vec{b_c}$ ортогональны.
      В-третьих, вектора $\vec{a_c}$, $\vec{b_c}$ и $\vec{с_c}$ представляют собой линейно независимую систему векторов.

      Параллелограмм задан также четырьмя векторами: $\vec{o_p}, \vec{a_p}, \vec{b_p}, \vec{c_p}$.
      Вектор $\vec{o_c}$ задает одну из вершин.
      Вектора $\vec{a_p}$, $\vec{b_p}$ и $\vec{с_p}$ задают три непараллельных ребра.
      Требование вытекает лишь одно: вектора $\vec{a_p}$, $\vec{b_p}$ и $\vec{с_p}$ должны образовывать линейно независимую систему.

      После получения всех параметров при создании объектов соответствующих классов производится проверка требований. 
      В случае, если требования не выполнены, программа выдаст сообщение об ошибке.
  \item
    \textbf{Преобразование базиса.}
      Перенесем начало координат в точку $\vec{o_c}$. 
      Вектора $\vec{a_c}$, $\vec{b_c}$ и $\vec{с_c}$ задают базис.
      Для этого базиса матрица перехода естественно определена:
      $N = \begin{pmatrix} {a_c} & {b_c} & c_{c} \end{pmatrix}$.
      Кроме того, введем новое скалярное произведение, определив матрицу Грама в новом базисе как единичную.
      Таким образом, в новом базисе цилиндр с внутреннотью задан системой неравенств 
      \begin{equation}\label{cnb}
          \begin{cases}
          x^2 + y^2   \le  1  \\
          0 \le  z   \le  1
          \end{cases}
      \end{equation}, 

      где $\begin{pmatrix} x \\ y \\ z \end{pmatrix}$ --- координаты некоторой точки в новом базисе.

      В новом базисе легко получим координаты для векторов, задающих параллелограмм.
      Для $\vec{o_p}$: 
      $\begin{pmatrix} x_o \\ y_o \\ z_o \end{pmatrix} =  N (o_p - o_c) $, где $o_p$ и $o_c$ --- коодинаты соответствующих векторов в исходном базисе.

      Для $\vec{a_p}$, $\vec{b_p}$ и $\vec{с_p}$:
      $\begin{pmatrix} x_a & x_b & x_c \\ y_a & y_b & y_c  \\ z_a & z_b & z_c \end{pmatrix} = 
      N \begin{pmatrix} a_p & b_p  & c_p \end{pmatrix}$

      Параллелограмму принадлежат точки $\vec{a}$, удовлетворяющие следующим условиям:
      \begin{equation}\label{pvec}
        \begin{cases}
          \vec{a} = \vec{o_p} + \alpha \vec{a_p} + \beta \vec{b_p} + \gamma \vec{c_p} \\
          0 \le  \alpha   \le  1 \\
          0 \le  \beta   \le  1 \\
          0 \le  \gamma   \le  1 
        \end{cases}
      \end{equation}

      Перепишем \ref{pvec} в координатном виде с координатами, заданными в новом базисе.
      \begin{equation}\label{pcoord}
        \begin{cases}
          \begin{pmatrix} x \\ y \\ z \end{pmatrix}            
          =
          \begin{pmatrix} x_o & x_a & x_b & x_c \\ y_o & y_a & y_b & y_c  \\ y_o & z_a & z_b & z_c 
          \end{pmatrix} \begin {pmatrix} 1 \\ \alpha \\ \beta \\ \gamma \end{pmatrix} \\
          0 \le  \alpha   \le  1 \\
          0 \le  \beta   \le  1 \\
          0 \le  \gamma   \le  1 
        \end{cases}
      \end{equation}

      Таким образом задача сводится к проверке наличия решения у систем \ref{cnb} и \ref{pcoord}.
  \item
    \textbf{Переход к задаче на гранях.}
    Поскольку и цилиндр, и параллелепипед --- тела выпуклые, цилиндр или вложен в параллелепипед, или пересекается с гранями параллелепипеда, или не имеет с ним общих точек вовсе.
    Грань параллелепипеда задается, например, следующей системой:
    \begin{equation}\label{planecoord}
      \begin{cases}
        \begin{pmatrix} x \\ y \\ z \end{pmatrix}            
        =
        \begin{pmatrix} x_o & x_a & x_b & x_c \\ y_o & y_a & y_b & y_c  \\ y_o & z_a & z_b & z_c 
        \end{pmatrix} \begin {pmatrix} 1 \\ \alpha \\ \beta \\ \gamma \end{pmatrix} \\
        0 \le  \alpha   \le  1 \\
        0 \le  \beta   \le  1 \\
        \gamma   =  1 
      \end{cases}
    \end{equation}
    

\end{enumerate}





\end{document}